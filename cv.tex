%!TEX TS-program = xelatex
%!TEX encoding = UTF-8 Unicode

\documentclass[10pt, a4paper]{cv}
\usepackage{fontspec} 
\usepackage{geometry}
\geometry{
	a4paper,
	left=25mm,
	top=20mm,
	right=25mm,
	bottom=20mm,
}
\pagenumbering{gobble}
\renewcommand*{\name}{\fontsize{24}{40}\mdseries\upshape}
% DOCUMENT
\begin{document}

\begin{center}
\name{Jason Wang}
\end{center}

Phone: \texttt{647-879-4660} \hfill
GitHub: \href{https://github.com/Vuwij}{https://github.com/Vuwij}\\
Email: \href{mailto:jiashen.wang@mail.utoronto.ca}{jiashen.wang@mail.utoronto.ca}\hfill
Website: \href{http://vuwij.github.io}{http://vuwij.github.io}

%%\hrule
\section*{Education}
\years{2015-2020}\textbf{Department of Electrical and Computer Engineering} \hfill University of Toronto\\
\emph{Bachelor of Computer Engineering, Minor in Robotics and Business} \small{\textsc{(3rd Year GPA 3.75)}}\\[0.1mm]
\small Dean's Honours List, NSERC USRA Research Award

%%\hrule
\section*{Technical Skills}
\textbf{Programming Languages:} C/C++, C\#, Matlab, Python, Mathematica, \LaTeX, Verilog/Assembly, HTML/CSS\\[0.2em]
\textbf{Software Frameworks:} ROS, Docker, OpenCV, Tensorflow, Unity Engine\\[0.2em]
\textbf{Relevant Courses:} Robot Modelling and Control, Inference Algorithms and Machine Learning, Control Systems, Digital Control, Algorithms and Data Structures, Computer Networks I, Operating Systems \\
\textbf{Languages:} English (Native), Chinese (Native), Japanese (Intermediate), French (Beginner)
%%\hrule
\section*{Work and Research Experience}\noindent
	\years{2018 -2019}\textbf{Robotics Software Engineering Intern} \hfill \href{https://www.rapyuta-robotics.com}{Rapyuta Robotics}
	\begin{itemize}
		\item Created a full simulation and integration testing system of a multi-robot warehouse delivery system. The robots coordinate with workers to fulfill orders retrieved from an inventory management system. Used ROS and C++ to write the simulation framework. Used docker and cloud infrastructure to simulate 30 robots communicating and fulfilling orders on the cloud.
		\item Working with the ALICA robot state machine engine, and used Google ORTools to optimize and simulate human movement to find the optimum human movement which reduces the cost.
	\end{itemize}

	\years{2018 May - }\textbf{Quadcopter Vision and Control Research (\href{https://github.com/utiasSTARS/hummingbird_ws}{Github})} \hfill \href{http://www.starslab.ca}{University of Toronto Institute for Aerospace Studies}
	\begin{itemize}
		\item Worked on the computer vision of a tailsitter drone with the capability to dock into other tailsitter drones in mid-flight. Used Apriltag tracking to obtain relative localization in order to plan the path for the drone to dock into the other drone. Supervised by \href{http://stars.utias.utoronto.ca/~jkelly/}{Professor Jonathan Kelly}.
	\end{itemize}

	\years{2017 May}\textbf{Medical Imaging Research Intern} \hfill Sunnybrook Hospital
	\begin{itemize}
		\item Created a \href{https://github.com/Vuwij/CPRadar}{Matlab} signal processing simulation for a new biomedical imaging device using ultra-wideband radar to assist heart surgery. Replicated results from a paper on radar imaging and presented results at University of Toronto's annual undergraduate research conference.
	\end{itemize}

\section*{Extracurricular Activities and Projects}

\subsection*{Engineering Student Groups}\noindent

	\years{2016-Now}\textbf{Founder and Software Lead} (\href{https://github.com/utra-robosoccer/soccer_ws} {Github}) \hfill \emph{\href{http://www.utrahumanoid.ca}{University of Toronto Robotics Association}}
	\begin{itemize}
		\item In 2017, formed a team of 20-30 members working on mechanical, electrical, software and control design of a humanoid soccer robot. Our faculty advisors are \href{http://stars.utias.utoronto.ca/~jkelly/}{Prof. Jonathan Kelly} and \href{http://www.utias.utoronto.ca/research/space-robotics/}{Prof. D’Eleuterio}.
		\item For robot control, used simple trajectory generation with constant linear feedback to keep the robot balanced. For computer vision, used OpenCV to integrate line detection with particle filtering to localize the robot. Used darknet for tracking the soccer ball. \href{https://github.com/utra-robosoccer/soccer_ws}{Source Code}
		\item Attended the 2018 robocup in Montreal as the only Canadian team in the small sized league consisting of purely undergraduate students. \href{http://utrahumanoid.ca}{Website}
	\end{itemize}
	
	\years{2016-17}\textbf{Team Lead} \hfill \emph{\href{http://hackeracademy.org}{Hacker Academy}}
	\begin{itemize}
		\item Created the challenge for the \href{https://www.facebook.com/groups/1126158850825448/}{DeepHealth Hackathon}, University of Toronto's first healthcare and machine learning hackathon. Around 80 graduate and undergraduate students attended the challenge.
	\end{itemize}

\end{document}
%!TEX TS-program = xelatex
%!TEX encoding = UTF-8 Unicode

\documentclass[11pt, a4paper]{cv}
\usepackage{fontspec} 
\usepackage{geometry}
\usepackage{xeCJK}
\usepackage{polyglossia}
\usepackage{babel}
\usepackage{fontspec}

\setdefaultlanguage{english}


% \usepackage{xeCJK}
% \setCJKmainfont{SimSun}
\geometry{
	a4paper,
	left=25mm,
	top=20mm,
	right=25mm,
	bottom=20mm,
}

\pagenumbering{gobble}
\renewcommand*{\name}{\fontsize{24}{40}\mdseries\upshape}
% DOCUMENT
\begin{document}

\begin{center}
\name{Jason Wang}
\end{center}


Phone: \texttt{} \hfill
GitHub: \href{https://github.com/Vuwij}{https://github.com/Vuwij}\\
Email: \href{mailto:vuwij1@gmail.com}{vuwij1@gmail.com}\hfill
Website: \href{http://vuwij.github.io}{http://vuwij.github.io}


%%\hrule
\section*{Selected Experience}\noindent
	\years{2020 May -\\ Current}\textbf{Robotics Software Team Lead} \hfill 	\href{https://www.rapyuta-robotics.com}{Rapyuta Robotics}
	\begin{itemize}
	\item Leading a team of 8 Software Engineers and Interns for the software development of a new warehouse robotic bin storage project and demoed successfully to customers
	\item Developed the overall software architecture and wrote most of the control, state machine, code for the robotic bin storage system.
	\item Used A* and some basic AI to optimize robot navigation in 3D structured space
	\item Dockerized, packaged and released IO-AMR robotics software to \href{https://www.hitachi.com/}{Hitachi}'s AI forklift system.
	\end{itemize}

	\years{2016 Sept - \\ Current}\textbf{Founder and Software Lead} (\href{https://github.com/utra-robosoccer/soccerbot} {Github}, \href{https://utra-robosoccer.github.io/}{Website}) \hfill \emph{\href{https://utra-robosoccer.github.io/}{University of Toronto Robosoccer Team}}
	\begin{itemize}
		\item In 2017, formed a team of 20-30 talented members for the \href{https://www.robocup.org/}{Robocup}'s humanoid kid-size competition with faculty advisors are \href{http://stars.utias.utoronto.ca/~jkelly/}{Prof. Jonathan Kelly} and \href{http://www.utias.utoronto.ca/research/space-robotics/}{Prof. D’Eleuterio}. The repository currently has over 100 stars.
		\item Created a walking engine using trajectory generation, classic control and IMU feedback using a PID feedback controller
		\item Used an ICP algorithm to get visual odometry from camera transformations of field lines detected using OpenCV and used it in a UKF to get optimal robot localization in a soccer field
		\item Attended in the 2018 humanoid kid-sized Robocup league in Montreal, 2021 humanoid kid-sized virtual Robocup league, and the 2022 humanoid kid-sized Robocup league in Thailand
	\end{itemize}

	\years{2018 May - \\2020 August}\textbf{Tailsitter Vision/Control Research Thesis (\href{https://github.com/utiasSTARS/PhoenixDrone}{Github})} \hfill \href{http://www.starslab.ca}{UTIAS}
	\begin{itemize}
	\item Funded by the NSERC USRA Summer Research Award, developed and extended calibration, localization, and navigation software of multi-tailsitter-type drones that docks onto other drones mid-flight in simulation. Used Apriltag bundle detection to obtain relative localization to another tailsitter drone. \\Supervisor: \href{http://stars.utias.utoronto.ca/~jkelly/}{Professor Jonathan Kelly}.
	\end{itemize}
	
\section*{Technical Skills}
\textbf{Programming Languages:} Python, C/C++, MATLAB, C\#, Mathematica, \LaTeX\\[0.2em]
\textbf{Software:} ROS, Docker, Django, OpenCV, Tensorflow, Pytorch, Unity Engine\\[0.2em]
\textbf{Languages:} English, Chinese, Japanese, Persian, Arabic, French

%%\hrule
\section*{Education}
\textbf{Masters of Engineering in Robotics and Systems Controls (Honours)}
\hfill University of Toronto\\
Courses: Convex Optimization, Robust and Optimal Control, State Estimation\\
Thesis: UKF Localization with Field Lines Observations for Humanoid Robotic Soccer\\


\textbf{Bachelor of Computer Engineering (with honours)} \hfill University of Toronto\\
\emph{with minor in Robotics and Business}\\
\small{
	3rd Year GPA: 3.75\hspace{1em}
	4th Year GPA: 3.91\hspace{1em}
	Dean's Honours List x8
}\\
Final Year Project: A Walking Acrobot - Winner of the Gordon Slemon Award (\$1000) for top 4th year project

%%\hrule



\end{document}
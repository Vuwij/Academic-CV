%!TEX TS-program = xelatex
%!TEX encoding = UTF-8 Unicode

\documentclass[10pt, a4paper]{cv}
\usepackage{fontspec} 
\usepackage{geometry}
\geometry{
	a4paper,
	left=25mm,
	top=20mm,
	right=25mm,
	bottom=20mm,
}
\pagenumbering{gobble}
\renewcommand*{\name}{\fontsize{24}{40}\mdseries\upshape}
% DOCUMENT
\begin{document}

\begin{center}
\name{Jason Wang}
\end{center}

Phone: \texttt{647-879-4660} \hfill
GitHub: \href{https://github.com/Vuwij}{https://github.com/Vuwij}\\
Email: \href{mailto:jiashen.wang@mail.utoronto.ca}{jiashen.wang@mail.utoronto.ca}\hfill
Website: \href{http://vuwij.github.io}{http://vuwij.github.io}

%%\hrule
\section*{Education}
\years{2020-\\current}\textbf{Department of Electrical and Computer Engineering} \hfill University of Toronto\\
\emph{Masters of Computer Engineering in Robotics and System Controls}\\

\years{Sept 2015-\\ April 2020}\textbf{Department of Electrical and Computer Engineering} \hfill University of Toronto\\
\emph{Bachelor of Computer Engineering with Honours, Minor in Robotics and Business}\\
\small{
	3rd Year GPA: 3.75\hspace{1em}
	4th Year GPA: 3.91\hspace{1em}
\small Dean's Honours List x7, NSERC USRA Summer Research Award

%%\hrule
\section*{Technical Skills}
\textbf{Programming Languages:} C/C++, Matlab, Python, C\#, Mathematica, \LaTeX, HTML/CSS\\[0.2em]
\textbf{Software Frameworks:} ROS, Docker, OpenCV, Tensorflow, Unity Engine\\[0.2em]
\textbf{Relevant Courses:} Robot Modeling and Control, Inference Algorithms and Machine Learning, Control Systems, Systems Control, Probabilistic Reasoning, Introduction to AI, Differential Geometry \\
\textbf{Languages:} English (Native), Chinese (Native), Japanese (Intermediate), French (Beginner), Farsi (Beginner)
%%\hrule
\section*{Work and Research Experience}\noindent
	\years{2020 May}\textbf{Software Engineer - Robotics} \hfill 	\href{https://www.rapyuta-robotics.com}{Rapyuta Robotics}
	\begin{itemize}
	\item Currently working with continuous integration of a automated forklift robot system.
	\end{itemize}

	\years{2018 May - \\2020 August}\textbf{Tailsitter Vision/Control Research Thesis (\href{https://github.com/utiasSTARS/PhoenixDrone}{Private Github})} \hfill \href{http://www.starslab.ca}{University of Toronto Institute for Aerospace Studies}
	\begin{itemize}
	\item Developed and extended calibration, localization, and navigation software of multi-tailsitter-type drones that docks onto other drones mid-flight in simulation. Used Apriltags bundle detection to obtain relative localization to another tailsitter drone. \\Supervisor: \href{http://stars.utias.utoronto.ca/~jkelly/}{Professor Jonathan Kelly}.
	\end{itemize}

	\years{2019 Sept - \\May}\textbf{A Walking Acrobot  Capstone Project (\href{https://github.com/Vuwij/acrobot}{Github})} \hfill \href{https://www.control.utoronto.ca/}{University of Toronto System Controls Laboratory}
	\begin{itemize}
		\item In a team of 4 built a two legged acrobotic robot controlled with a single DC motor that uses guidance via virtual holonomic constraints involving impacts.
		\item Succeeded in taking 1.5 steps on the actual built robot and 10 steps in simulation.
		\item Won the Gordon Slemon Award (\$1000) for excellence in Engineering Design\\ Supervisor: \href{https://www.control.utoronto.ca/~maggiore/}{Professor Manfredi Maggiore}.
	\end{itemize}

	\years{2018 Sept - \\ 2019 Sept}\textbf{Software Engineering Intern - Robotics} \hfill \href{https://www.rapyuta-robotics.com}{Rapyuta Robotics}
	\begin{itemize}
		\item Created a full simulation and integration testing system of a multi-robot warehouse delivery system. The robots coordinate with workers to fulfill orders retrieved from an inventory management system. Used ROS and C++ to write the simulation framework. Used docker and cloud infrastructure to simulate 30 robots communicating and fulfilling orders on the cloud.
		\item Worked with the ALICA robot state machine engine, and used Google ORTools to optimize and simulate human movement to find the optimum human movement which reduces movement time.
	\end{itemize}

%	\years{2017 May -\\ 2017 Sept}\textbf{Student Researcher} \hfill \href{https://sunnybrook.ca/content/?page=schulich-heart-centre}{Sunny Schulich Heart Program}
%	\begin{itemize}
%		\item Worked on a new ultra-wideband radar device to assist with MRI imaging \href{https://github.com/Vuwij/CPRadar}{(Matlab)}
%		\item Presented research findings at the \href{https://www.facebook.com/UnERD2019/?ref=py_c}{University of Toronto Undergraduate Research Conference (UnERD)}
%	\end{itemize}
%
%	\years{2016 May -\\ 2017 Dec}\textbf{Software Engineering Intern} \hfill \href{https://www.bell.ca/}{Bell Business Markets}
%	\begin{itemize}
%		\item Worked on deployment and operations of a new telephone banking system in Java
%	\end{itemize}
\pagebreak
\section*{Extracurricular Activities and Projects}

\subsection*{Engineering Student Groups}\noindent

	\years{2016-Now}\textbf{Founder and Software Lead} (\href{https://github.com/utra-robosoccer/soccer_ws} {Github}) \hfill \emph{\href{http://www.utrahumanoid.ca}{University of Toronto Robotics Association}}
	\begin{itemize}
		\item In 2017, formed a team of 20-30 talented members working on mechanical, electrical, software and control design of a humanoid soccer robot. Our faculty advisors are \href{http://stars.utias.utoronto.ca/~jkelly/}{Prof. Jonathan Kelly} and \href{http://www.utias.utoronto.ca/research/space-robotics/}{Prof. D’Eleuterio}.
		\item For robot control, used simple trajectory generation with constant IMU feedback to keep the robot balanced. For computer vision, used OpenCV to integrate line detection with particle filtering to localize the robot. Used darknet for tracking the soccer ball. \href{https://github.com/utra-robosoccer/soccer_ws}{Source Code}
		\item Attended the 2018 robocup in Montreal as the only Canadian team in the small sized league consisting of purely undergraduate students. Attempting to attend the 2020 Robocup Asia Pacific in Tokyo. \href{http://utrahumanoid.ca/our-project/}{Website}
	\end{itemize}
	
	\years{2016-17}\textbf{Team Lead} \hfill \emph{\href{https://www.facebook.com/hackeracademy/}{Hacker Academy}}
	\begin{itemize}
		\item Created the challenge for the \href{https://www.facebook.com/groups/1126158850825448/}{DeepHealth Hackathon}, University of Toronto's first healthcare and machine learning hackathon. Around 80 graduate and undergraduate students attended the challenge.
	\end{itemize}

\end{document}